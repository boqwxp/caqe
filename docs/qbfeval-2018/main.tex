\documentclass[11pt,twocolumn]{article}

\title{CAQE at QBFEval 2018}
\author{Leander Tentrup, Saarland University}
\date{}

\usepackage{amsmath}

\newcommand{\caqe}{\text{CAQE}}

\begin{document}
  \maketitle
  
\section{Introduction}

This paper presents the QBF solver $\caqe$~\cite{conf/fmcad/RabeT15} as submitted to QBFEval 2018.
$\caqe$ aims to be a high performance QBF solver by carefully mixing orthogonal solving techniques and using optimized data structures.

\section{Major Improvements}

\paragraph{New Implementation.}

The new version of $\caqe$ is implemented in the programming language Rust.
The previous C implementation accumulated much cruft originating to the initial implementation from 2013.
The new implementation optimizes data structures and avoided bottlenecks that we discovered over the years.

\paragraph{Expansion and Strong Unsat Refinement}

As last years version, $\caqe$ implements two additional refinement methods as described in~\cite{conf/cav/Tentrup17}.
\emph{Expansion refinement} does partial expansion for the innermost quantifier alternation similar to RAReQS.
\emph{Strong unsat refinement} improves on the standard disjunctive refinement by excluding a conjunction subsumed clauses.

\paragraph{Abstraction Optimization}

\paragraph{Preprocessing}

\paragraph{QDIMACS Output}


\section{Thanks}

This work was supported by the European Research Council (ERC) Grant OSARES (No. 683300).


\bibliographystyle{plain}
\bibliography{main}

\end{document}