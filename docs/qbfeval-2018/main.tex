\documentclass[11pt,twocolumn]{article}

\title{CAQE at QBFEval 2018}
\author{Leander Tentrup, Saarland University \and Markus N. Rabe, University of California, Berkeley}
\date{}

\usepackage{amsmath}
\usepackage{mathpazo}
\usepackage{Alegreya}

\newcommand{\caqe}{\text{CAQE}}

\begin{document}

\maketitle
  
\section{Introduction}

This paper presents the QBF solver $\caqe$~\cite{conf/fmcad/RabeT15} as submitted to QBFEval 2018.
$\caqe$ aims to be a high performance QBF solver by carefully mixing orthogonal solving techniques and using optimized data structures.

\section{Major Improvements}

\paragraph{New Implementation.}

The new version of $\caqe$ is implemented in the programming language Rust.
The previous C implementation accumulated much cruft originating back to the initial implementation from 2014.
The new implementation optimizes data structures and avoided bottlenecks that we discovered over the years.
As the underlying SAT solver, we use CryptoMiniSat~\cite{conf/sat/SoosNC09} in version 5.5.

\paragraph{Preprocessing.}

Preprocessing is an important factor in the performance of $\caqe$.
We submitted two configurations of $\caqe$, one using Bloqqer~\cite{conf/cade/BiereLS11} and one using HQSPre~\cite{conf/tacas/WimmerRM017}.

\paragraph{Expansion and Strong Unsat Refinement.}

As last years version, $\caqe$ implements two additional refinement methods~\cite{conf/cav/Tentrup17}.
\emph{Expansion refinement} does partial expansion for the innermost quantifier alternation similar to RAReQS.
\emph{Strong unsat refinement} improves on the standard disjunctive refinement by excluding a conjunction subsumed clauses.

\paragraph{Tree-shaped Quantifier Prefix.}
$\caqe$ does recursion on the quantifier prefix of the QBF.
From the linear quantifier prefix, we try to reconstruct a tree-shaped prefix by applying the mini-scoping rules. 

\paragraph{Abstraction Optimization.}

We implemented an extension of the abstraction minimization techniques~\cite{conf/sat/BalabanovJSMB16}.
For universal abstractions, we use the universal variable as $t$-literal if it is the only universal variable in the clause (and the outermost quantified variable in clause).
For both polarities, we do only generate one (abstraction) clause if two clauses are equal with respect to variables bound $\leq$ current level.
We also implemented refinement literal subsumption, but in our experiments this did not improve solving time.

\paragraph{QDIMACS Output.}

$\caqe$ now supports the QDIMACS output format.
Using the partial assignments from Bloqqer~\cite{conf/date/SeidlK14}, $\caqe$ can provide complete assignments even in case of preprocessing.


\section{Acknowledgments}

This work was supported by the European Research Council (ERC) Grant OSARES (No. 683300).


\bibliographystyle{plain}
\bibliography{main}

\end{document}