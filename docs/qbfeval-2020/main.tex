\documentclass[11pt,twocolumn]{article}

\title{CAQE and QuAbS at QBFEval 2020}
\author{Leander Tentrup, Saarland University}
\date{}

\usepackage{amsmath}
\usepackage{mathpazo}
\usepackage{Alegreya}
\usepackage{url}
\usepackage{cite}

\newcommand{\caqe}{\text{CAQE}}
\newcommand{\dcaqe}{\text{dCAQE}}
\newcommand{\quabs}{\text{QuAbS}}

\begin{document}

\maketitle
  
\section{Introduction}

This paper gives an overview of the submissions to QBFEval 2019.
The prenex CNF solver $\caqe$~\cite{conf/fmcad/RabeT15} and the prenex non-CNF solver $\quabs$~\cite{journals/corr/Tentrup16}.
We refer to~\cite{journals/jsat/Tentrup19} for a detailed system description.


\subsection{$\caqe$}

The solver $\caqe$ (version $4.0$) is based on the \emph{clausal abstraction} algorithm~\cite{conf/fmcad/RabeT15,conf/ijcai/JanotaM15}\footnote{called \emph{clause selection} in~\cite{conf/ijcai/JanotaM15}} for solving QBF.
The underlying idea of clausal abstraction is to assign variables, where the assignment order is determined by the quantifier prefix until either all clauses are satisfied or there is a set of clauses that cannot be satisfied at the same time.
The effect of assignments, i.e., whether they satisfy a clause, is abstracted into one bit of information per clause, and this information is communicated through the quantifier prefix.
The fundamental data structure of the algorithm is an abstraction, a propositional formula for each maximal block of quantifiers, that, given the valuation of outer variables, generates candidate assignments for the variables bound at this quantifier block.
In case this candidate is refuted by inner quantifiers, the returned counterexample is excluded in the abstraction.
Thus, the clausal abstraction algorithm uses ideas of search-based solving~\cite{series/faia/GiunchigliaMN09}, and counterexample guided abstraction refinement (CEGAR) algorithms~\cite{conf/cav/ClarkeGJLV00}.
A proof-theoretic analysis of the clausal abstraction approach~\cite{conf/cav/Tentrup17} has shown that the refutation proofs correspond to the (level-ordered) $Q$-resolution calculus~\cite{journals/iandc/BuningKF95}.

The implementation is written in Rust and uses CryptoMiniSat~\cite{conf/sat/SoosNC09} in version 5.0.1 as the underlying SAT solver.
The source code of $\caqe$ is available at \url{https://github.com/ltentrup/caqe}.
We submitted three configurations of $\caqe$, one using only HQSPre~\cite{conf/tacas/WimmerRM017} as preprocessor, one configuration using both, Bloqqer~\cite{conf/cade/BiereLS11} and HQSPre, and one configuration that produces QDIMACS outputs using the partial assignments from Bloqqer~\cite{conf/date/SeidlK14}.


\subsection{$\quabs$}

$\quabs$ is based on the extension of clausal abstraction to formulas in negation normal form~\cite{journals/corr/Tentrup16,conf/sat/Tentrup16}.
The implementation is written in C++ and uses CryptoMiniSat~\cite{conf/sat/SoosNC09} in version 5.5 as the underlying SAT solver.
The source code of $\quabs$ is available at \url{https://github.com/ltentrup/quabs}.
In addition to $\quabs$, we submitted a configuration of $\caqe$ that transforms the circuit representation and its negation to CNF.


\section{Acknowledgments}

This work was partially supported by the German Research Foundation (DFG) as part of the Collaborative Research Center ``Foundations of Perspicuous Software Systems’' (TRR 248, 389792660) and by the European Research Council (ERC) Grant OSARES (No. 683300).


\bibliographystyle{plain}
\bibliography{main}

\end{document}